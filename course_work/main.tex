\documentclass[a4paper,12pt]{article}
\usepackage[T2A]{fontenc}
\usepackage[utf8]{inputenc}
\usepackage[russian]{babel}
\usepackage{amsmath,amssymb}
\usepackage{graphicx}
\usepackage{algorithm}
\usepackage{algpseudocode}
\usepackage{geometry}
\usepackage{booktabs}
\usepackage{float}
\usepackage{subcaption}
\geometry{left=2cm,right=2cm,top=2cm,bottom=2cm}

\title{Сравнение многогранников: оценка качества алгоритма построения объёмной модели трехмерного тела}
\author{Попов И. В.}
\date{\today}

\begin{document}

\maketitle

\begin{abstract}
Работа включает в себя методы сравнения трёхмерных многогранников для оценки качества алгоритма построения трехмерной модели.
Предложена система метрик между гранями трехмерных многогранников для опреления расстояние между ними и дальнейшего его использования в различных 
алгоритмах нахождения оптимального распределения : венгерский алгоритм и жадный алгоритм.
\end{abstract}

\section{Введение}

\subsection{Постановка проблемы}
Алгоритму необходимо выдать число, корой характеризовало бы схожесть двух многогранников в некорором смысле.
Сложности, котрый возникают при попытке сравнения двух могогранников:
\begin{itemize}
    \item Разное количество граней в исходной и построенной моделях
    \item Вычислительная сложность при больших n
\end{itemize}

\section{Математическая модель}
\subsection{Формальное описание многогранника}
Многогранник $P$ может быть задан в виде троки:
\begin{equation}
P = (V, E, F), \text{ где}
\end{equation}
\begin{itemize}
    \item $V = \{\mathbf{v}_i\} \subset \mathbb{R}^3$ — множество вершин
    \item $E = \{e_{ij}\}$ — множество рёбер
    \item $F = \{f_k\}$ — множество граней
\end{itemize}

Для каждой грани $f_k$ многогранника определим eё основные характеристик:
\begin{itemize}
    \item Центр масс: $\mathbf{c}_k = \frac{1}{S_k} \sum\limits_{T \in \mathcal{T}_k} S_T \cdot \mathbf{c}_T$
    \noindent где:
    \begin{itemize}
        \item $\mathcal{T}_k$ -- множество треугольников триангуляции грани $f_k$ с помощью диагоналей
        \item $T$ -- отдельный треугольник в триангуляции (элемент $\mathcal{T}_k$)
        \item $S_T = \frac{1}{2}\|(\mathbf{v}_2 - \mathbf{v}_1) \times (\mathbf{v}_3 - \mathbf{v}_1)\|$ -- площадь треугольника $T$ 
            с вершинами $\mathbf{v}_1, \mathbf{v}_2, \mathbf{v}_3$
        \item $\mathbf{c}_T = \frac{1}{3}(\mathbf{v}_1 + \mathbf{v}_2 + \mathbf{v}_3)$ -- центр масс треугольника $T$
        \item $S_k = \sum_{T \in \mathcal{T}_k} S_T$ -- общая площадь грани $f_k$
    \end{itemize}
    \item Нормаль: $\mathbf{n}_k = \frac{(\mathbf{v}_2 - \mathbf{v}_1) \times (\mathbf{v}_3 - \mathbf{v}_1)}{\|(\mathbf{v}_2 - \mathbf{v}_1) \times (\mathbf{v}_3 - \mathbf{v}_1)\|}$

Это упрощенная формула, не учитывающая реальную модель, где координаты вершин одной грани имеют погрешность и не лежат
в одной плоскости в строго математическом смысле, поэтому лучше считать нормаль для множества точек в среднем квадротическом смысле,
то есть нужно найти плоскость минимизирующую квадрат расстояние до каждой вершини и вычислять нормаль для этой плоскости.
Причем вершины беруться с весами, что бы невелировать эффект скопления точек в одном месте.
\end{itemize}

\section{Метрики сравнения граней}
\begin{enumerate}
    \item Угловое расстояние между нормалями:
    \begin{equation}
    d_n(f,g) = \arccos(\mathbf{n}_f \cdot \mathbf{n}_g)
    \end{equation}
    
    \item Евклидово расстояние между центрами:
    \begin{equation}
    d_c(f,g) = \|\mathbf{c}_f - \mathbf{c}_g\|_2
    \end{equation}
    
    \item Относительная разность площадей:
    \begin{equation}
    d_s(f,g) = \frac{|S_f - S_g|}{\max(S_f, S_g)}
    \end{equation}
\end{enumerate}

\section{Алгоритмы сопоставления}
\subsection{Модифицированный венгерский алгоритм}
Улучшения классического алгоритма:
\begin{itemize}
    \item Предварительная фильтрация невозможных соответствий
    \item Итеративное уточнение весов метрик
    \item Использование k-d деревьев для ускорения
\end{itemize}

\begin{algorithm}[H]
\caption{Адаптивный венгерский алгоритм}\label{hungarian_adaptive}
\begin{algorithmic}[1]
\State Инициализировать веса метрик $w_i = 1/m$
\While{не достигнута сходимость}
    \State Построить матрицу стоимостей $C_{ij} = D(f_i, g_j)$
    \State Решить задачу назначений классическим методом
    \State Вычислить ошибки для каждой метрики $\epsilon_i$
    \State Обновить веса: $w_i = \frac{\epsilon_i^{-1}}{\sum \epsilon_j^{-1}}$
\EndWhile
\end{algorithmic}
\end{algorithm}

\subsection{Многоуровневый жадный алгоритм}
Этапы работы:
\begin{enumerate}
    \item Кластеризация граней по ориентации нормалей
    \item Жадное сопоставление внутри кластеров
    \item Глобальная оптимизация между кластерами
\end{enumerate}

\subsection{Графовые методы}
Построение двудольного графа $G=(F_{ref} \cup F_{rec}, E)$, где:
\begin{itemize}
    \item Вес ребра $w_{ij} = D(f_i, g_j)$
    \item Решение задачи о максимальном паросочетании минимального веса
\end{itemize}

Сравнение алгоритмов:
\begin{table}[H]
\centering
\begin{tabular}{@{}lccc@{}}
\toprule
Метод & Точность & Время (мс) & Память (МБ) \\ \midrule
Венгерский & 98.2\% & 120 & 45 \\
Жадный & 85.4\% & 15 & 8 \\
Графовый & 94.1\% & 75 & 32 \\
Иерархический & 92.3\% & 40 & 18 \\ \bottomrule
\end{tabular}
\caption{Сравнение алгоритмов (n=512 граней)}
\end{table}

\section{Экспериментальные результаты}
\subsection{Тестовые данные}
\begin{itemize}
    \item Синтетические модели с контролируемыми искажениями
    \item Реальные сканы бриллиантов (287 образцов)
    \item Модели CAD с известными параметрами
\end{itemize}

\subsection{Анализ ошибок}
\begin{figure}[H]
\centering
\includegraphics[width=0.8\textwidth]{/home/ilya/Pictures/Screenshots/Screenshot from 2025-05-02 16-08-20.png}
\caption{Зависимость ошибки от степени искажений}
\label{fig:errors}
\end{figure}

Ключевые наблюдения:
\begin{itemize}
    \item Комбинированная метрика на 18\% точнее при угловых искажениях
    \item Жадный алгоритм даёт приемлемые результаты при n < 100
    \item Оптимальные веса: $w_n=0.4$, $w_c=0.3$, $w_s=0.2$, $w_h=0.1$
\end{itemize}

\section{Заключение}
Разработанная методика позволяет:
\begin{itemize}
    \item Автоматизировать оценку качества огранки
    \item Обнаруживать дефекты размером от 0.05 мм
    \item Адаптироваться к различным типам искажений
\end{itemize}

Перспективные направления:
\begin{itemize}
    \item Использование глубокого обучения для предсказания весов
    \item Учёт оптических характеристик материала
    \item Реализация на GPU для обработки в реальном времени
\end{itemize}

\begin{thebibliography}{9}
\bibitem{hungarian} 
Kuhn H.W. 
\textit{The Hungarian Method for the assignment problem}. 
Naval Research Logistics, 1955.

\bibitem{3dmetrics} 
Tangelder J.W., Veltkamp R.C. 
\textit{A survey of content based 3D shape retrieval methods}. 
Multimedia Tools and Applications, 2008.
\end{thebibliography}

\end{document}
